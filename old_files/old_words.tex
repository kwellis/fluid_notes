\section{Old Words}

Another aspect that is made difficult by the sonic velocity is calculating when the $dE_{te}$ will cross the zero axis. When the fluid is operating sub sonic, or under a Mach value of 0.8, the relationship of $dE_{te}$ versus pressure is linear. As the fluid speeds up above mach 0.8, the slope of $dE_{te}$ against pressure starts to level off. When the fluid is operating at Mach One, the value of $\frac{dE_{te}}{dP}$ is zero. This is a known behavior in compressible fluid mechanics, where many relationships will flip in value once the fluid is super sonic \cite{fluids_white}. A derivation that analytically validates this behavior is show in the appendix and summarized in equation \eqref{dete_dp}. 

In figure \ref{fig:entr_four}, the suction pressure starts at 890 psig, the pressure is gradually dropped in increments until \dete equals zero. At the point \dete equals zero the calculated throat entrance pressure is 500 psig. Figure \ref{fig:entr_four} has one additional complication because the \dete is crossing the zero axis while also hitting the sonic velocity. This point can be thought of as the lowest theoretical suction pressure a specific jet pump can operate for a given reservoir. Attempting to operate the suction below this point will cause an accumulation in the throat entrance, driving the suction pressure back up. The other observation that should be noticed from the figure is the change in slope of \dete against pressure in the last plot. The change in slope occurs as the fluid transitions from subsonic to transonic. Transonic flow is commonly recognized to start at Mach 0.8 and continue until Mach 1.2. During this period, the fluid transitions from subsonic to supersonic flow. Though super sonic flow cannot physically occur inside an oil jet pump, the calculations still capture the correct theoretical behavior. A well documented product of sonic flow above Mach 1.0 is that the behavior of fluid relationships are flipped. The easiest of this being that in subsonic flow, a restriction in flow area caused the fluid velocity to increase. In sonic velocity, the behavior is reversed, and a restriction causes the fluid to slow down. Getting into the details of this is beyond the scope of this report. What is important is that the behavior is properly reflected in the numerical analysis generated in this report and shown in figure 3. Additionally a derivation is provided in appendix (reference) that confirms this behavior from first principles. The resulting equation that is of importance for this report is equation \eqref{dete_dp}.

\begin{equation}
\frac{dE_{te}}{dp} = \frac{1}{\rho}(1 - \Mach^2)
\label{dete_dp}
\end{equation}

\section{Fluids Overview}

The reservoir fluid flows to the suction, while the power fluid flows to the nozzle inlet. Both the reservoir fluid and power fluid experience an increase in velocity as they are respectively constricted through the throat entrance and nozzle tip. The increase in velocity creates a drop in pressure, allowing the two fluids to reach an equilibrium pressure. In the throat momentum is transferred from the power fluid to the reservoir fluid and mixing occurs. The cross sectional area is increased in the diffuser, reducing the velocity and increasing the pressure. The fluid leaves the diffuser and flows to surface.  jet pump and will be starting point of discussion.