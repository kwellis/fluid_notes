\documentclass{article}
% \usepackage{graphicx} % Required for inserting images
\usepackage{amsmath}
\usepackage[style=ieee, url=false]{biblatex}
\usepackage{parskip} % adds a skipped paragraph
\addbibresource{jpump.bib} %Import the bibliography file

\newcommand\Mach{\mbox{\textit{Ma}}}  % Mach number
\newcommand\Tee{\mbox{\textit{TEE}}}  % Throat Entry Equation

\title{Jet Pump Equation}
\author{Kaelin Ellis}
\date{February 2024}

\begin{document}

\maketitle

\section{Introduction}

A jet pump is made up of four different components. The jet pump components are a nozzle, enterance, throat and diffuser. The starting point for three of the four equations is the differential form of the energy equation.

\begin{equation}
dE = \frac{dP}{\rho} + VdV + gdz
\end{equation}

In a jetpump the difference in height is negligible and the term can be omitted.

\begin{equation}
    dE = \frac{dP}{\rho} + VdV
\end{equation}

For the differential energy equation to have physical meaning, the difference in energy term must equal zero. Any number other than zero is a violation of the law of conservation of energy. For the jet pump enterance the $dE_{te}$ term is calculated at different conditions in an attempt to solve the jet pump equations.

\section{Nozzle}

The nozzle is the easiest component to analytically find the solution for. Traditionally water is used as the power fluid, which is an incompressible fluid.

\begin{equation}
\int{\frac{dP}{\rho}} + \int{VdV} = 0
\end{equation}

\begin{equation}
\int_{ni}^{nz} \frac{dP}{\rho} + \int_{ni}^{nz} VdV = 0
\end{equation}

\end{document}